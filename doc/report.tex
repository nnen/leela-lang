%%%%%%%%%%%%%%%%%%%%%%%%%%%%%%%%%%%%%%%%%%%%%%%%%%%%%%%%%%%%%%%%%%%%%%%
%% PREAMBLE
%%%%%%%%%%%%%%%%%%%%%%%%%%%%%%%%%%%%%%%%%%%%%%%%%%%%%%%%%%%%%%%%%%%%%%%
\documentclass[10pt,a4paper]{article}

\usepackage[czech]{babel}
\usepackage[utf8]{inputenc}
%\usepackage{amsmath,amsfonts,amsthm,amssymb}
\usepackage{fullpage} % 1 inch okraje
\usepackage{hyperref} % hypertextové odkazy
%\usepackage{graphicx} % vkládání grafiky
%\usepackage{float,wrapfig}
\usepackage{fancyhdr}
%\usepackage{lastpage}
\usepackage{listings}
\usepackage{longtable}

\newcommand{\paperTitle}{Implementace programovacího jazyka Leela}
\newcommand{\paperType}{Semestrální práce}
\newcommand{\paperClass}{BI-PJP}
\newcommand{\paperDueDate}{}
\newcommand{\paperAuthorName}{Jan Milík}
\newcommand{\paperAuthorEmail}{milikjan@fit.cvut.cz}

\newenvironment{codedoc}[1]
	{ \vspace{4pt} \noindent \texttt{#1} \par \addtolength{\leftskip}{\parindent} }
	{ \addtolength{\leftskip}{-1.0\parindent} }

%\topmargin=-0.45in      %
%\evensidemargin=0in     %
%\oddsidemargin=0in      %
%\textwidth=6.5in        %
%\textheight=9.0in       %
%\headsep=0.25in         %

\pagestyle{fancy}
\lhead{\paperAuthorName}
\rhead{\paperClass: \paperTitle}
\cfoot{Strana \thepage{}}
\renewcommand\headrulewidth{0.4pt} %
\renewcommand\footrulewidth{0.4pt} %

\title{\vspace{1in}\paperTitle\\
\small{\paperClass -- \paperType}\vspace{0.5in}}
\author{\paperAuthorName\\
\small\texttt{{\href{mailto:\paperAuthorEmail}{\paperAuthorEmail}}}}

% Settings for the listings package
\lstset{
	breaklines=true,
	extendedchars=\true,
	%inputencoding=utf8x
}

%%%%%%%%%%%%%%%%%%%%%%%%%%%%%%%%%%%%%%%%%%%%%%%%%%%%%%%%%%%%%%%%%%%%%%%
%% DOKUMENT
%%%%%%%%%%%%%%%%%%%%%%%%%%%%%%%%%%%%%%%%%%%%%%%%%%%%%%%%%%%%%%%%%%%%%%%
\begin{document}
% Titulní strana
\maketitle
\newpage

% Obsah
\tableofcontents
\newpage

%%%%%%%%%%%%%%%%%%%%%%%%%%%%%%%%%%%%%%%%%%%%%%%%%%%%%%%%%%%%%%%%%%%%%%%
%% ÚVOD
%%%%%%%%%%%%%%%%%%%%%%%%%%%%%%%%%%%%%%%%%%%%%%%%%%%%%%%%%%%%%%%%%%%%%%%
\section{Úvod}
\label{sec:uvod}

Jazyk Leela vychází, a měl by být nadmnožinou, jazyka Míla, jehož autorem
je prof. Muller. Jeho název byl zvolen tak, aby se vyslovoval ``Líla''
(Míla + Lambda = Líla), ale zároveň je kulturní referencí. Její odhalení
ponechávám jako cvičení čtenáři.

\section{Zásobníkový počítač}
\label{sec:pocitac}

\subsection{Zásobník}

Zásobníkový počítač má oddělený zásobník pro aktivační rámce (dále `call
stack') a zásobník pro výpočty. Aktivační rámce jsou reprezentovány
objekty (typu \texttt{ActivationFrame}), a každý obsahuhe tato data:

\begin{itemize}
\item Svůj vlastní datový zásobník - instrukce operující nad zásobníkem
      implicitně používají datový zásobník aktivačního rámce na vrcholu call stacku,
\item referenci na funkční objekt, jehož zavoláním daný rámec vznikl (pro debugování),
\item pole referencí na objekty typu \texttt{Variable}, které reprezentují
      proměnné; v tomto poli se nacházejí všechny proměnné dostupné z dané funkce
	 (kromě konstant a věstavených symbolů, které jsou zvláštní případ) a jsou trojího
	 druhu:
	 \begin{enumerate}
	 \item Volné proměnné - toto jsou proměnné, které pocházejí z kontextu, ve kterém
	       byla funkce vytvořena. Referenci na tyto proměnné (uzávěru) si s sebou nese
		  funkční objekt, dokud není zavolán. Při zavolání, ještě před předáním parametrů
		  a naalokováním lokálních proměnných funkční objekt předá reference na volné
		  proměnné aktivačnímu rámci.
	 \item Parametry funkce - jsou předávány hodnotou z aktivačního rámce, který
	       byl na vrcholu call stacku v momentě, kdy byl vytvářen nový rámec.
		  Instrukce \texttt{CALL} má argument, který říká, kolik hodnot má být ze
		  stacku vyzvednuto a předáno funkčnímu objektu při volání. Funkční objekt
		  má informaci o tom, kolik argumentů přijímá a pokud je mu předán jiný počet,
		  vyvolá runtime chybu. Tuto kontrolu by bylo jen velice těžké provádět při
		  překladu, protože funkce jsou v Leele objekty první třídy a tak libovolný
		  symbol může být v libovolnou chvíli svázán s libovolnou funkcí.
	 \item Lokální proměnné funkce - jsou alokovány pomocí instrukce \texttt{ALLOC} a
	       jsou to nově vytvořené proměnné, které jsou inicializovány na hodnotu typu
		  \texttt{None}.
	 \end{enumerate}
\item adresu intrukce, která bude vykonaná v dalším kroku (instruction pointer).
\end{itemize}

\subsection{Bytekód}

Bytekód je v paměti reprezentován jako jednoduché pole bytů a třída
\texttt{Bytecode} udržuje ukazatel na aktuální pozici v tomto poli a pomáhá
interpretovat byty na aktuální pozici jako instrukce, nebo textové řetězce.
Intrukce mají žádný, nebo jeden argument. Kód instrukce má šířku 8 bitů a
argument 32 bitů. Argument může být interpretován jako celé číslo se znaménkem,
nebo bez znaménka v závislosti na dané instrukci. Pokud daná instrukce
argument nemá, argument se do bytekódu vůbec neukládá, což má za
následek složitější načítání instruckí a práci s adresami, ale na
druhou stranu to redukuje velikost bytekódu.

\subsection{Instrukce}

\begin{center}
\begin{tabular}{ l l l }
Datový typ & \verb"stdint.h" ekvivalent & Popis \\
\hline
\texttt{Integer}  & \verb"int32_t" & signed 32 bit integer \\
\texttt{UInteger} & \verb"uint32_t" & unsigned 32 bit integer \\
\texttt{Address}  & \verb"uint32_t" & unsigned 32 bit integer \\
\end{tabular}
\end{center}

Seznam není kompletní; vynechal jsem některé instrukce jako např. aritmetické
operace nebo operace pro porovnávání v zájmu ochrany našich lesů.

\begin{longtable}{ l l l p{0.5\textwidth} }
\hline
Opcode & Argument & Typ argumentu & Význam \\
\hline
\endhead
\texttt{STOP} & & & Ukončuje okamžitě běh programu nezávisle na stavu zásobníkového počítače. \\
\texttt{RETURN} & & & Vyzvedne hodnotu z dat. zásobníku aktuálního rámce, vyhodí aktuální rámec z call stacku a vrátí vyzvednutou hodnotu na dat. zásobník akt. rámce na vrchu call stacku. \\
\texttt{CALL} & & & Vyzvedne hodnotu z vrchu dat. zásobníku, pokusí se jí konvertovat na hodnotu typu \texttt{Callable} a zavolá její metodu \texttt{Value::call}. \\
\texttt{JUMP} & adresa & \texttt{Address} & Nastaví instruction counter aktuálního rámce na danou adresu. \\
\verb"JUMP_IF" & adresa & \texttt{Address} & Vyzvedne hodnotu ze stacku, konvertuje ji na typ \texttt{Boolean} a pokud má hodnotu \texttt{true}, nastaví instruction counter na danou adresu. \\
\texttt{PUSH} & number & \texttt{Integer} & Vloží na zásobník hodnotu typu \texttt{Number}. \\
\verb"PUSH_STRING" & adresa & \texttt{Address} & Přečte z dané adresy z bytekódu textový řetězec, vytvoří nový objekt typu \texttt{String} a vloží ho na vrch dat. zásobníku. \\
\texttt{POP} & & & Vyhodí hodnotu z vrcholu dat. zásobníku a zapomene ji. Použité pro "zapomínání" návratových hodnot funkcí, pokud nejsou ničemu přiřazované a nejsou použité v nějakém výpočtu. \\
\verb"LOAD_CONST" & index & \texttt{UInteger} & Najde v poli konstant hodnotu s daným indexem a vloží ji na vrchol dat. zásobníku. \\
\verb"DEF_CONST" & & & Zvedne hodnotu z vrcholu dat. zásobníku a vloží ji na konec pole konstant, a definuje tak vlastně novou konstantu. \\
\texttt{LOAD} & index & \texttt{UInteger} & Vloží na vrchol dat. zásobníku hodnotu proměnné, která má v aktuálním kontextu daný index. \\
\texttt{STORE} & index & \texttt{UInteger} & Vyzvedne ze zásobníku hodnotu a přiřadí ji proměnné s daným indexem. \\
\texttt{ALLOC} & počet & \texttt{UInteger} & Přidá na konec pole proměnných aktuálního akt. rámce daný počet nových proměnných. Proměnné jsou inicializované na hodnotu typu \texttt{None}. \\ 
\verb"STORE_R[0-9]" & adresa & \texttt{Address} & Uloží hodnotu svého argumentu do registru 0-9. \\
\verb"MAKE_FUNCTION" & adresa & \texttt{Address} & Vyzvedne ze zásobníku hodnotu, pokusí se ji konvertovat na typ \texttt{Number}, vytvoří nový funkční objekt, nastaví jeho adresu kódu na hodnotu atributu instrukce a jeho počet argumentů na hodnotu vyzvednutou ze zásobníku. \\
\verb"LOAD_CLOSURE" & index & \texttt{UInteger} & Vyhledá na zasobníku nejbližší rámec, jehož funkce má stejnou adresu jako adresa uložená v registru 0 a z tohoto rámce zkopíruje referenci na proměnnou s daným indexem do uzávěry funkčního objektu na vrchu dat. zásobníku. \\
\hline
\end{longtable}

\section{EBNF jazyka Leela}
\label{sec:ebnf}

\begin{lstlisting}[identifierstyle=\textbf]
program = { var-declaration | const-declaration } compound-statement

var-declaration = `var' identifier { `,' identifier } `;'
const-declaration = `const' identifier `=' const-expression { `,' identifier `=' const-expression } `;'

compound-statement = `begin' statement { `;' statement } `end'
statement = compound-statement
          | assignment-or-call
          | `if' expression `then' statement [ `else' statement ]
          | `while' expression `do' statement
          | `print' right-value
          | e

assignment-or-call = identifier { `.' identifier | `[' right-value `]' } ( `:=' right-value | `(' expression-list `)' )

right-value = `function' `(' [ identifier { `,' identifier } ] `)' `:' { var-declaration } statement
            | `lambda' `(' [ identifier { `,' identifier } ] `)' `:' right-value
            | expression

expression = sum { ( `=' | `<>' | `<' | `>' | `<=' | `>=' ) sum }
sum = [ `-' ] term { ( `-' | `+' ) term }
term = factor { ( `*' | `/' ) factor }
factor = primary-expression { `[' right-value `]' | `.' identifier | `(' param-list `)' }
primary-expression = identifier | const-expression | `(' right-value `)'

param-list = [ right-value { `,' right-value } ]

\end{lstlisting}

\section{LL(1) (téměř) gramatika}

Podobně jako vzorový jazyk Míla, i v této gramatice existuje FIRST-FOLLOW
konflikt u neterminálního symbolu \textit{ElseStmt}. Překladač tento problém
řeší tak, že terminální symbol \textit{else} je jednoduše přijatý tím
``nejhlubším'', nebo, chcete-li, nejbližší neterminálním symbolem
\textit{ElseStmt}. Výsledek je takový, že jazyk přijímá věty, u kterých si
programátor nemusí být jistý, jak budou překladačem interpretovány. Nicméně
překlad je deterministický.

\begin{longtable}{ l c l }
	Program          & $\rightarrow$ & Preamble CompoundStmt \\
	
	Preamble         & $\rightarrow$ & ConstDecl Preamble \\
	Preamble         & $\rightarrow$ & VarDecl Preamble \\
	Preamble         & $\rightarrow$ & $\epsilon$ \\
	
	ConstDecl        & $\rightarrow$ & const ident = ConstExpr ConstDeclRest ; \\
	
	ConstDeclRest    & $\rightarrow$ & , ident = ConstExpr ConstDeclRest \\
	ConstDeclRest    & $\rightarrow$ & $\epsilon$ \\
	
	ConstExpr        & $\rightarrow$ & number \\
	ConstExpr        & $\rightarrow$ & string \\

	VarDecl          & $\rightarrow$ & var ident VarDeclRest ; \\
	
	VarDeclRest      & $\rightarrow$ & , ident VarDeclRest \\
	VarDeclRest      & $\rightarrow$ & $\epsilon$ \\

	Statement        & $\rightarrow$ & CompoundStmt \\
	Statement        & $\rightarrow$ & Assignment \\
	Statement        & $\rightarrow$ & print RValue \\
	Statement        & $\rightarrow$ & if Expression then Statement ElseStmt \\
	Statement        & $\rightarrow$ & while Expression do Statement \\
	Statement        & $\rightarrow$ & return RValue \\ 
	Statement        & $\rightarrow$ & $\epsilon$ \\

	CompoundStmt     & $\rightarrow$ & begin Statement StatementRest end \\
	
	StatementRest    & $\rightarrow$ & ; Statement StatementRest \\
	StatementRest    & $\rightarrow$ & $\epsilon$ \\

	ElseStmt         & $\rightarrow$ & else Statement \\
	ElseStmt         & $\rightarrow$ & $\epsilon$ \\

	Assignment       & $\rightarrow$ & ident AssignVar \\
	
	AssignVar        & $\rightarrow$ & := RValue \\
	AssignVar        & $\rightarrow$ & . ident AssignItem \\
	AssignVar        & $\rightarrow$ & [ RValue ] AssignItem \\
	AssignVar        & $\rightarrow$ & ( ParamList ) \\

	AssignItem       & $\rightarrow$ & := RValue \\
	AssignItem       & $\rightarrow$ & . ident AssignItem \\
	AssignItem       & $\rightarrow$ & [ RValue ] AssignItem \\
	AssignItem       & $\rightarrow$ & ( ParamList ) \\

	RValue           & $\rightarrow$ & Expression \\
	RValue           & $\rightarrow$ & function ( ArgNameList ) : FunctionPreamble Statement \\
	RValue           & $\rightarrow$ & lambda ( ArgNameList) : RValue \\
	
	FunctionPreamble & $\rightarrow$ & VarDecl FunctionPreamble \\
	FunctionPreamble & $\rightarrow$ & $\epsilon$ \\

	Expression       & $\rightarrow$ & Sum ExpressionRest \\
	
	ExpressionRest   & $\rightarrow$ & = Sum ExpressionRest \\
	ExpressionRest   & $\rightarrow$ & <> Sum ExpressionRest \\
	ExpressionRest   & $\rightarrow$ & < Sum ExpressionRest \\
	ExpressionRest   & $\rightarrow$ & > Sum ExpressionRest \\
	ExpressionRest   & $\rightarrow$ & <= Sum ExpressionRest \\
	ExpressionRest   & $\rightarrow$ & >= Sum ExpressionRest \\
	ExpressionRest   & $\rightarrow$ & $\epsilon$ \\

	Sum              & $\rightarrow$ & Term SumRest \\
	Sum              & $\rightarrow$ & - Term SumRest \\

	SumRest          & $\rightarrow$ & + Term SumRest \\
	SumRest          & $\rightarrow$ & - Term SumRest \\
	SumRest          & $\rightarrow$ & $\epsilon$ \\
	
	Term             & $\rightarrow$ & Factor TermRest \\
	
	TermRest         & $\rightarrow$ & * Factor TermRest \\
	TermRest         & $\rightarrow$ & / Factor TermRest \\
	TermRest         & $\rightarrow$ & $\epsilon$ \\

	Factor           & $\rightarrow$ & PrimaryExpr PortfixOp \\
	
	PrimaryExpr      & $\rightarrow$ & ident \\
	PrimaryExpr      & $\rightarrow$ & ConstExpr \\
	PrimaryExpr      & $\rightarrow$ & ( RValue ) \\

	PostfixOp        & $\rightarrow$ & [ Expression ] PostfixOp \\
	PostfixOp        & $\rightarrow$ & ( ParamList ) PostfixOp \\
	PostfixOp        & $\rightarrow$ & $\epsilon$ \\

	ParamList        & $\rightarrow$ & RValue ParamListRest \\
	ParamList        & $\rightarrow$ & $\epsilon$ \\

	ParamListRest    & $\rightarrow$ & , RValue ParamListRest \\
	ParamListRest    & $\rightarrow$ & $\epsilon$ \\
	
	ArgNameList      & $\rightarrow$ & ident ArgNameListRest \\
	ArgNameList      & $\rightarrow$ & $\epsilon$ \\
	
	ArgNameListRest  & $\rightarrow$ & , ident ArgNameListRest \\
	ArgNameListRest  & $\rightarrow$ & $\epsilon$ \\
\end{longtable}

\section{Překladač}
\label{sec:prekladac}

\subsection{Architektura}

Leela se skládá ze tří částí:
\begin{enumerate}
\item Leela překladač, který přeloží zdrojový kód jazyka Leela do textového mezijazyka, Lasm,
\item Leela assembler, který zkompiluje mezijazyk Lasm do bytekódu, Leelac
\item a Leela machine, zásobníkový počítač, který umí přečíst bytekód a vykonat ho.
\end{enumerate}

\subsection{Implementace}

Překladač je implementován pomocí motedy rekurzivního sestupu. Překladač
nevytváří žádný AST a neprovádí žádné optimalizace. Jako pomocné struktury
používá
\begin{itemize}
\item tabulku kontextů (každý kontext má svojí vlastní tabulku symbolů),
\item tabulku textových řetězců,
\item tabulku věstavených symbolů (hardcoded)
\item a zapisovač mezijazyka (obsahuje tabulku bloků výstupu, které odpovídají
      jednotlivým zanořeným lambda výrazům a anonymním metodám - v kódu jsou
      do sebe jednotlivé kontexty zanořené, ale na výstupu jsou seřazené linárně
      za sebou).
\end{itemize}
Je reprezentován třídou \texttt{Parser} a jeho instanční metody implementují
jednotlivá pravidla gramatiky a sémantické akce. 

Metoda pro přijetí neterminálního symbolu má signaturu
\begin{lstlisting}
Ref<Object> Parser::methodName(Ref<Object> inherited, vector<Ref<Object> > siblings);
\end{lstlisting}
kde návratová hodnota reprezentuje syntetizovaný atribut neterminálního
symbolu, argument \texttt{inherited} je syntetizovaný symbol rodičovského symbolu
a atribut \texttt{siblings} je pole syntetizovaných atributů předcházejících
symbolů (terminálních a neterminálních) uvnitř rodičovského neterminálního
symbolu. Toto schéma pro zjednodušení předpokládá, že zděděné atributy symbolu
jsou jednoduše syntetizované atributy jeho rodiče a předcházejících sourozenců
v syntaktickém stromu.

Tyto metody jsou automaticky generované z textového souboru
(\verb"leela_grammar.txt"), který obsahuje definici gramatiky v jednoduchém
formátu. Generátorem je skript napsaný v jazyce Python, který pomocí knihovny
\texttt{ply} (yacc + flex pro Python) přečte definici gramatiky a přeloží do
jednoduchého AST a poté z těchto dat za pomocí knihovny \texttt{jinja2}
vygeneruje část hlavičkového souboru pro třídu \texttt{Parser} a C++ soubor s
implementacemi výše zmíněných metod.

Sémantické akce jsou implementovány ručně. Jejich signature je
\begin{lstlisting}
void Parser::name(Ref<Object> inherited,
                  vector<Ref<Object> > siblings, 
                  vector<Ref<Object> >& match,
                  Ref<Object>& result);
\end{lstlisting}
kde \texttt{match} je seznam syntetizovaných atributů symbolů, které v daném
neterminálu předcházely volané sémantické akci. Parametr \texttt{result} je
reference na proměnnou, která drží syntetizovaný atribut neterminálu, který
sémantickou akci vyvolal. Pomocí tohoto parametru může sémantická akce změnit
hodnotu syntetizováného atributu svého neterminálu. Parametry \texttt{inherited}
a \texttt{siblings} jsou obdobné jako u metod pro neterminální symboly.

\end{document}

