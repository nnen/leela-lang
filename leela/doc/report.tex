%%%%%%%%%%%%%%%%%%%%%%%%%%%%%%%%%%%%%%%%%%%%%%%%%%%%%%%%%%%%%%%%%%%%%%%
%% PREAMBLE
%%%%%%%%%%%%%%%%%%%%%%%%%%%%%%%%%%%%%%%%%%%%%%%%%%%%%%%%%%%%%%%%%%%%%%%
\documentclass[10pt,a4paper]{article}

\usepackage[czech]{babel}
\usepackage[utf8]{inputenc}
%\usepackage{amsmath,amsfonts,amsthm,amssymb}
%\usepackage{fullpage} % 1 inch okraje
\usepackage{hyperref} % hypertextové odkazy
%\usepackage{graphicx} % vkládání grafiky
%\usepackage{float,wrapfig}
\usepackage{fancyhdr}
%\usepackage{lastpage}
\usepackage{listings}

\newcommand{\paperTitle}{Implementace programovacího jazyka Leela}
\newcommand{\paperType}{Semestrální práce}
\newcommand{\paperClass}{BI-PJP}
\newcommand{\paperDueDate}{}
\newcommand{\paperAuthorName}{Jan Milík}
\newcommand{\paperAuthorEmail}{milikjan@fit.cvut.cz}

\newenvironment{codedoc}[1]
	{ \vspace{4pt} \noindent \texttt{#1} \par \addtolength{\leftskip}{\parindent} }
	{ \addtolength{\leftskip}{-1.0\parindent} }

%\topmargin=-0.45in      %
\evensidemargin=0in     %
\oddsidemargin=0in      %
\textwidth=6.5in        %
%\textheight=9.0in       %
%\headsep=0.25in         %

\pagestyle{fancy}
\lhead{\paperAuthorName}
\rhead{\paperClass: \paperTitle}
\cfoot{Strana \thepage{}}
\renewcommand\headrulewidth{0.4pt}                                      %
\renewcommand\footrulewidth{0.4pt}                                      %

\title{\vspace{1in}\paperTitle\\
\small{\paperClass -- \paperType}\vspace{0.5in}}
\author{\paperAuthorName\\
\small\texttt{{\href{mailto:\paperAuthorEmail}{\paperAuthorEmail}}}}

% Settings for the listings package
\lstset{
	breaklines=true
}

%%%%%%%%%%%%%%%%%%%%%%%%%%%%%%%%%%%%%%%%%%%%%%%%%%%%%%%%%%%%%%%%%%%%%%%
%% DOKUMENT
%%%%%%%%%%%%%%%%%%%%%%%%%%%%%%%%%%%%%%%%%%%%%%%%%%%%%%%%%%%%%%%%%%%%%%%
\begin{document}
% Titulní strana
\maketitle
\newpage

% Obsah
\tableofcontents
\newpage

%%%%%%%%%%%%%%%%%%%%%%%%%%%%%%%%%%%%%%%%%%%%%%%%%%%%%%%%%%%%%%%%%%%%%%%
%% ÚVOD
%%%%%%%%%%%%%%%%%%%%%%%%%%%%%%%%%%%%%%%%%%%%%%%%%%%%%%%%%%%%%%%%%%%%%%%
\section{Úvod}
\label{sec:uvod}

Jazyk Leela vychází, a měl by být nadmnožinou, jazyka Míla, jehož autorem
je prof. Muller. Jeho název byl zvolen tak, aby se vyslovoval ``Líla''
(Míla + Lambda = Líla), ale zároveň je kulturní referencí. Její odhalení
ponechávám jako cvičení čtenáři.

\section{EBNF jazyka Leela}
\label{sec:ebnf}

\begin{lstlisting}
program = preamble compound-statement

preamble = { var-declaration | const-declaration }
var-declaration = `var' identifier { `,' identifier } `;'
const-declaration = `const' identifier `=' const-expression { `,' identifier `=' const-expression } `;'

compound-statement = `begin' statement { `;' statement } `end'
statement = compound-statement
          | assignment-or-call
          | `if' expression `then' statement [ `else' statement ]
          | `while' expression `do' statement
          | 'print' right-value
          | e

assignment-or-call = identifier { `.' identifier | `[' right-value `]' } ( `:=' right-value | `(' expression-list `)' )

table-literal = `{' table-item { ',' table-item } `}'
table-item    = right-value [ ':' right-value ]
              | e

\end{lstlisting}

\end{document}

